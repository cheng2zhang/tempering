\documentclass[11pt]{article}
\usepackage{amsmath}
\usepackage{graphicx}
\usepackage{xcolor}
\usepackage{hyperref}

\hypersetup{
    colorlinks,
    linkcolor={red!30!black},
    citecolor={blue!50!black},
    urlcolor={blue!80!black}
}

\begin{document}

\title{Velocity scaling after a temperature transition in a molecular-dynamics-based simulated tempering}
\author{}
\date{\vspace{-10ex}}
\maketitle

Simulated tempering\cite{lyubartsev1992, marinari1992},
including that variant based on a continuous temperature set-up\cite{zhang2010},
is based on an expanded ensemble framework in which
the temperature is a random variable.
%
Unlike the Monte Carlo counterpart,
the molecular dynamics (MD) implementation of simulated tempering
is trickier in that the temperature influences
the sampling in the coordinate-space
via the velocities and the thermostat.
%
More precisely, an MD-based simulated tempering
is conducted in an extended configuration space
spanned by the coordinates, $\mathbf r$,
velocities, $\mathbf v$,
and (inverse) temperature, $\beta$.
%
The sampling in this space is realized
by three types of moves:
%
\begin{enumerate}
  \item
    Time evolution of the coordinates, $\mathbf r$, and velocities, $\mathbf v$, by Newton's equation.

  \item
    Scaling and perturbation of the velocities, $\mathbf v$, by the thermostat.

  \item
    Change of the reference temperature, $\beta$, of the thermostat
    under fixed the coordinates, $\mathbf r$.
\end{enumerate}
%
The first two moves constitute
a constant-temperature MD.
%
The last, the temperature move,
is the addition of simulated tempering.
%
Usually, such a temperature move is based on
the potential energy, $U(\mathbf r)$,
of the current coordinates, $\mathbf r$.
%
For example,
the Metropolis rule is to accept an unbiased transition
from $\beta$ to $\beta'$ by probability
%
\begin{equation}
  A(\beta \to \beta')
  =
  \min\left\{
    1,
    \frac{
      \exp[ -\beta' \, U(\mathbf r) ] / Z(\beta')
    }
    {
      \exp[ -\beta \, U(\mathbf r) ] / Z(\beta)
    }
  \right\}
  .
  \label{eq:Tmove_Metropolis}
\end{equation}
%
Alternatively, for a continuous temperature set-up,
we can drive the temperature by a Langevin equation\cite{zhang2010},
%
\begin{equation}
  \frac{ d } { dt }
  \left(
  \frac{ 1 } { \beta }
  \right)
  =
  U(\mathbf r)
  -
  \langle U \rangle_\beta
  +
  \frac{ \sqrt 2  } { \beta } \, \xi
  ,
  \label{eq:Tmove_Langevin}
\end{equation}
%
where $\xi$ is a unit-variant Gaussian white noise.


We want to show in this note that after a successful
temperature transition from $\beta$ to $\beta'$,
the velocities, $\mathbf v$, must be multiplied by
a factor of $\sqrt{ \beta / \beta' }$.
%
The reason is that the temperature move
given by Eq. \eqref{eq:Tmove_Metropolis}
or \eqref{eq:Tmove_Langevin}
is based on the potential energy, $U(\mathbf r)$, alone,
%as a function of the coordinates, $\mathbf r$, only.
%
and it only qualifies the coordinates, $\mathbf r$,
to enter the new temperature.
%
The velocities, $\mathbf v$, are left out of
the screening process
and have to be oriented to fit into
the distribution at the new temperature.
%
Below, we shall formulate the above argument
mathematically in terms of detailed balance.


\section{Detailed Balance}


We can design a correct temperature transition
by satisfying detailed balance,
which ensures the cancellation
of the mutual probability flows between
the two configurations before and after
a successful temperature transition.

Let us first show that a temperature transition without
velocity scaling does not work.
%
In this case, the configuration before the temperature transition
is given by $\mu = (\mathbf r, \mathbf v, \beta)$,
and that after the transition by
$\nu = (\mathbf r, \mathbf v, \beta')$.
%
We want to check if the detailed balance condition
between $\mu$ and $\nu$ holds:
%
\begin{equation}
  p_\mu \, A(\mu \to \nu)
  =
  p_\nu \, A(\nu \to \mu)
  .
\label{eq:db_novs}
\end{equation}
%
Assuming a flat overall temperature distribution, $w(\beta) = c$,
$p_\mu$ is just the canonical distribution at temperature, $\beta$:
%
\begin{equation}
  p(\mu)
  =
  c \,
  \frac{
    \exp\left[
       -\beta \, U(\mathbf r)
    \right]
  }
  {
    Z(\beta)
  }
  d \mathbf r
  \frac{
    \exp\left(
      -\beta \, \frac{1}{2} \mathbf v \cdot \mathbf M \cdot \mathbf v
    \right)
  }
  {
    Y(\beta)
  }
  d \mathbf v
  ,
\label{eq:pmu}
\end{equation}
%
and $p_\nu$ that at temperature $\beta'$:
%
\begin{equation}
  p(\nu)
  =
  c \,
  \frac{
    \exp\left[
       -\beta' \, U(\mathbf r)
    \right]
  }
  {
    Z(\beta')
  }
  d \mathbf r
  \frac{
    \exp\left(
      -\beta' \, \frac{1}{2} \mathbf v \cdot \mathbf M \cdot \mathbf v
    \right)
  }
  {
    Y(\beta')
  }
  d \mathbf v
  .
\notag
%\label{eq:pnu}
\end{equation}
%
Here, $Z(\beta)$ and $Y(\beta)$
are the respective normalization factors in
the coordinates and velocities spaces,
%
\begin{align}
  Z(\beta)
  &=
  \int \exp\left[ -\beta \, U(\mathbf r) \right] \, d\mathbf r
  ,
  \notag
\\
  Y(\beta)
  &=
  \int \exp\left(
      -\beta \, \frac{1}{2} \mathbf v \cdot \mathbf M \cdot \mathbf v
  \right) \, d\mathbf v
  =
  \left(
    \frac{ 2 \, \pi } { \beta }
  \right)^{ N_f / 2 }
  \frac{
    1
  }
  {
    \sqrt{ \det \mathbf M }
  }
  ,
  \label{eq:Ybeta}
\end{align}
%
with $\mathbf M$ being the diagonal mass matrix,
and $N_f$ being the number of degrees of freedom.
%
Note that we have included $d\mathbf r$ and $d\mathbf v$
in the definitions of $p(\mu)$ and $p(\nu)$
to remind us the phase-space volume element.

Now using Eqs. \eqref{eq:pmu} and \eqref{eq:Tmove_Metropolis}
on the left-hand side of Eq. \eqref{eq:db_novs}, we get
%
\begin{equation}
  c \,
  \min\left\{
    \frac{
      \exp\left[ -\beta \, U(\mathbf r) \right]
    }
    {
      Z(\beta)
    }
    ,
    \frac{
      \exp\left[ -\beta' \, U(\mathbf r) \right]
    }
    {
      Z(\beta')
    }
  \right\}
  \, d \mathbf r \,
  \frac{
    \exp\left(
      -\beta \, \frac{1}{2} \mathbf v \cdot \mathbf M \cdot \mathbf v
    \right)
  }
  {
    Y(\beta)
  }
  d \mathbf v
  .
\label{eq:pflow_left}
\end{equation}
%
But for the right-hand side, we get
%
\begin{equation}
  c \,
  \min\left\{
    \frac{
      \exp\left[ -\beta \, U(\mathbf r) \right]
    }
    {
      Z(\beta)
    }
    ,
    \frac{
      \exp\left[ -\beta' \, U(\mathbf r) \right]
    }
    {
      Z(\beta')
    }
  \right\}
  \, d \mathbf r \,
  \frac{
    \exp\left(
      -\beta' \, \frac{1}{2} \mathbf v \cdot \mathbf M \cdot \mathbf v
    \right)
  }
  {
    Y(\beta')
  }
  d \mathbf v
  .
\label{eq:pflow_left}
\end{equation}
%
We see that the two differ by the last factor.
%
So the transition without velocity scaling is
not necessarily correct.


\section{Velocity scaling}

Next, let us see how velocity scaling saves the day.
%
The key is to establish a detailed balance
between $\mu = (\mathbf r, \mathbf v, \beta)$
and the \emph{velocity-scaled} configuration
$\nu' = (\mathbf r, \mathbf v', \beta')$,
with
%
\begin{equation}
  \mathbf v'
  =
  \sqrt{ \frac{ \beta } { \beta' } }
  \, \mathbf v
  ,
\label{eq:vscale}
\end{equation}
%
being the scaled velocity after a successful transition
from $\beta$ to $\beta'$.
%
Conversely,
after the transition from $\beta'$ to $\beta$,
$\mathbf v'$ is transformed to
%
$$
  \mathbf v''
  =
  \sqrt \frac{ \beta' } { \beta } \,
  \mathbf v'
  =
  \mathbf v,
$$
which means the velocity scaling scheme also pairs
$\nu'$ back to $\mu$.


To verify detailed balance,
we have for the forward flow
%
\begin{equation}
  p(\mu) \, A(\mu \to \nu')
  =
  \min\left\{
    \frac{
      e^{ -\beta \, U(\mathbf r) }
    }
    {
      Z(\beta)
    }
    ,
    \frac{
      e^{ -\beta' \, U(\mathbf r) }
    }
    {
      Z(\beta')
    }
  \right\}
  \, d \mathbf r \,
  \frac{
    \exp\left(
      -\beta \, \frac{1}{2} \mathbf v \cdot \mathbf M \cdot \mathbf v
    \right)
  }
  {
    Y(\beta)
  }
  d \mathbf v
  ,
\label{eq:pflow_left_vs}
\end{equation}
%
which is same as Eq. \eqref{eq:pflow_left}.
%
For the backward flow,
we have
$$
  p(\nu')
  =
  \frac{
    \exp\left[
       -\beta' \, U(\mathbf r)
    \right]
  }
  {
    Z(\beta')
  }
  d \mathbf r
  \frac{
    \exp\left(
      -\beta' \, \frac{1}{2} \mathbf v' \cdot \mathbf M \cdot \mathbf v'
    \right)
  }
  {
    Y(\beta')
  }
  d \mathbf v'
  ,
$$
such that
%
\begin{equation}
  p(\nu') \, A(\nu' \to \mu)
  =
  \min\left\{
    \frac{
      e^{ -\beta \, U(\mathbf r) }
    }
    {
      Z(\beta)
    }
    ,
    \frac{
      e^{ -\beta' \, U(\mathbf r) }
    }
    {
      Z(\beta')
    }
  \right\}
  \, d \mathbf r \,
  \frac{
    \exp\left(
      -\beta' \, \frac{1}{2} \mathbf v' \cdot \mathbf M \cdot \mathbf v'
    \right)
  }
  {
    Y(\beta')
  }
  d \mathbf v'
  .
\label{eq:pflow_right_vs}
\end{equation}
%
But this time,
the last factors from Eqs. \eqref{eq:pflow_left_vs}
and \eqref{eq:pflow_right_vs} agree,
because we have, from
Eq. \eqref{eq:vscale},
%
$$
  \beta \, \frac{1}{2} \mathbf v \cdot \mathbf M \cdot \mathbf v
  =
  \beta' \, \frac{1}{2} \mathbf v' \cdot \mathbf M \cdot \mathbf v'
  ,
$$
%
and from Eq. \eqref{eq:Ybeta},
%
$$
  \frac{ d \mathbf v } { Y(\beta) }
  =
  \frac{
    \sqrt { \frac { \beta' } { \beta } }^{ N_f }
    \, d \mathbf v'
  }
  {
    \left(
      \frac{ 2 \, \pi } { \beta }
    \right)^{ N_f / 2 }
    \frac{ 1 } { \sqrt { \det \mathbf M } }
  }
  =
  \frac{ d \mathbf v' } { Y(\beta') }
  .
$$
%
In other words, detailed balance is satisfied
and with velocity scaling,
we have achieved a correct implementation of temperature transitions.


\section{Discussions}


Although our demonstration above
assumes Eq. \eqref{eq:Tmove_Metropolis} as a special case,
it can be generalized to any potential-energy-based
temperature moves,
as long as a \emph{balance} condition
can be established.\footnote{
  For Eq. \eqref{eq:Tmove_Langevin},
  the balance condition amounts to checking if
  the corresponding Fokker-Planck equation
  for $\tau = 1/\beta$
  permits the desired distribution, $q(\tau)$,
  in the stationary condition.
  %
  The Fokker-Planck equation reads
  $$
    \frac{ \partial q(\tau) } { \partial \tau }
    =
    -\frac{ \partial } { \partial \tau }
    \left\{
      [U - \langle U \rangle_\beta] \, q(\tau)
      -
      \frac{ \partial } { \partial \tau }
      \left[ \tau^2 \, q(\tau) \right]
    \right\}
    .
  $$
  In the stationary condition, we get
  $ [U - \langle U \rangle_\beta] \, q(\tau)
      -
      \frac{ \partial } { \partial \tau }
      \left[ \tau^2 \, q(\tau) \right] = 0$,
  whose solution is
  $$
    \tau^2 \, q(\tau) \propto
    \frac{ \exp[-\beta \, U(\mathbf r)] } { Z(\beta) }.
  $$
  By changing back to the variable $\beta$
  using the conservation of probability,
  $ p(\beta) \, |d\beta| = q(\tau) \, |d\tau|$,
  we get the desired distribution of $\beta$:
  $$
    p(\beta)
    =
    q(\tau) \left| \frac{ d\tau } { d\beta } \right|
    \propto
    \frac{ \exp[-\beta \, U(\mathbf r)] } { Z(\beta) }
    .
  $$
}
%
\begin{equation}
  \int
    \frac{ \exp\left[ -\beta \, U(\mathbf r) \right] }
         { Z( \beta ) }
    A(\beta \to \beta') \, d\beta
  =
    \frac{ \exp\left[ -\beta' \, U(\mathbf r) \right] }
         { Z( \beta' ) }
  .
\label{eq:beta_balance}
\end{equation}
%

One way to avoid velocity scaling is to use the force scaling technique.
%
The idea is to treat temperature, $\beta$, as a parameter that
modifies the potential energy as
%
\begin{equation}
  U_\mathrm{eff}(\mathbf r; \beta)
  =
  \frac{ \beta } { \beta_0 } \,
  U(\mathbf r)
  .
\label{eq:Ueff}
\end{equation}
%
We can then simulate the system under the effective potential
with a thermostat tuned to a constant temperature, $\beta_0$.
%
For a configuration, $\mu = (\mathbf r, \mathbf v, \beta)$,
the distribution is given by
%
\begin{align}
  p^\mathrm{fs}(\mu)
  &=
  \frac{
    \exp\left\{
      -\beta_0 \,
      \left[
          U_\mathrm{eff}(\mathbf r; \beta)
          +\frac{1}{2} \mathbf v \cdot \mathbf M \cdot \mathbf v
      \right]
    \right\}
  }
  {
    Z(\beta) \, Y(\beta)
  }
  d \mathbf r \,
  d \mathbf v
  \notag
\\
  &=
  \frac{
    \exp\left[
       -\beta \, U(\mathbf r)
    \right]
  }
  {
    Z(\beta)
  }
  d \mathbf r
  \frac{
    \exp\left(
      -\beta_0 \, \frac{1}{2} \mathbf v \cdot \mathbf M \cdot \mathbf v
    \right)
  }
  {
    Y(\beta_0)
  }
  d \mathbf v
  .
\label{eq:pmu_fs}
\end{align}
%
In this way, the velocities always feel the temperature $\beta_0$;
whereas the coordinates feel the variable temperature $\beta$.
%
One can show that the distribution given by Eq. \eqref{eq:pmu_fs}
satisfies the detailed balance condition, Eq. \eqref{eq:db_novs}
without velocity scaling,
and the algorithm is correct.
%
To implement this technique, we scale the force by
a factor of $(\beta / \beta_0)$, because from Eq. \eqref{eq:Ueff},
we have
%
$$
\mathbf F_\mathrm{eff} = \frac{ \beta } { \beta_0 } \, \mathbf F.
$$
However, nothing needs to be done on the thermostat or the velocities.



Another way to avoid the velocity scaling is to replace
the potential energy $U(\mathbf r)$ by the total energy,
$E(\mathbf r, \mathbf v)$,
in Eq. \eqref{eq:Tmove_Metropolis} or \eqref{eq:Tmove_Langevin}.
%
For example, Eq. \eqref{eq:Tmove_Metropolis} becomes
%
\begin{equation}
  A_E(\beta \to \beta')
  =
  \min\left\{
    1,
    \frac{
      \exp[ -\beta' \, E(\mathbf r, \mathbf v) ] /
      [Z(\beta') \, Y(\beta')]
    }
    {
      \exp[ -\beta \, E(\mathbf r, \mathbf v) ] /
      [Z(\beta) \, Y(\beta)]
    }
  \right\}
  .
\label{eq:Tmove_Metropolis_E}
\end{equation}
%
This, however, is less efficient,
as it decreases the distribution overlap between the two temperatures,
as shown in Fig. \ref{fig:overlap}.
%
For example, with an ideal-gas system, $U(\mathbf r) \equiv 0$,
the acceptance probability from Eq. \eqref{eq:Tmove_Metropolis}
is always unity, but that from Eq. \eqref{eq:Tmove_Metropolis_E}
is less.


\begin{figure}[h]
\begin{center}
  \makebox[\linewidth][c]{
    \includegraphics[angle=0, width=\linewidth]{fig/overlap.pdf}
  }
  \caption{
    \label{fig:overlap}
    Velocity scaling improves the overlap
    between the distributions between the two temperatures,
    making the temperature transition more efficient.
  }
\end{center}
\end{figure}


\bibliographystyle{abbrv}
\bibliography{simul}
\end{document}

